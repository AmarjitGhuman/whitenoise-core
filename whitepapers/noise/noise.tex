\documentclass[11pt]{scrartcl} % Font size
%%%%%%%%%%%%%%%%%%%%%%%%%%%%%%%%%%%%%%%%%
% Wenneker Assignment
% Structure Specification File
% Version 2.0 (12/1/2019)
%
% This template originates from:
% http://www.LaTeXTemplates.com
%
% Authors:
% Vel (vel@LaTeXTemplates.com)
% Frits Wenneker
%
% License:
% CC BY-NC-SA 3.0 (http://creativecommons.org/licenses/by-nc-sa/3.0/)
%
%%%%%%%%%%%%%%%%%%%%%%%%%%%%%%%%%%%%%%%%%

%----------------------------------------------------------------------------------------
%	PACKAGES AND OTHER DOCUMENT CONFIGURATIONS
%----------------------------------------------------------------------------------------

\usepackage{amsmath, amsfonts, amsthm} % Math packages
\usepackage{hyperref}
\usepackage{comment}
\usepackage{amsmath}
\usepackage{amssymb}
\usepackage{bbm}
\usepackage{algpseudocode, algorithm, algorithmicx}
\usepackage[toc,page]{appendix}
\usepackage[user,titleref]{zref}

\hypersetup{
   colorlinks=true,
   citecolor=red,
   linkcolor=cyan,
   filecolor=magenta,
   urlcolor=blue,
 }

\usepackage{url}
\usepackage{listings} % Code listings, with syntax highlighting

\usepackage[english]{babel} % English language hyphenation

\usepackage{graphicx} % Required for inserting images
\graphicspath{{Figures/}{./}} % Specifies where to look for included images (trailing slash required)

\usepackage{booktabs} % Required for better horizontal rules in tables

\numberwithin{equation}{section} % Number equations within sections (i.e. 1.1, 1.2, 2.1, 2.2 instead of 1, 2, 3, 4)
\numberwithin{figure}{section} % Number figures within sections (i.e. 1.1, 1.2, 2.1, 2.2 instead of 1, 2, 3, 4)
\numberwithin{table}{section} % Number tables within sections (i.e. 1.1, 1.2, 2.1, 2.2 instead of 1, 2, 3, 4)

\setlength\parindent{0pt} % Removes all indentation from paragraphs

\usepackage{enumitem} % Required for list customisation
\setlist{noitemsep} % No spacing between list items

%----------------------------------------------------------------------------------------
%	DOCUMENT MARGINS
%----------------------------------------------------------------------------------------

\usepackage{geometry} % Required for adjusting page dimensions and margins

\geometry{
	paper=a4paper, % Paper size, change to letterpaper for US letter size
	top=2.5cm, % Top margin
	bottom=3cm, % Bottom margin
	left=3cm, % Left margin
	right=3cm, % Right margin
	headheight=0.75cm, % Header height
	footskip=1.5cm, % Space from the bottom margin to the baseline of the footer
	headsep=0.75cm, % Space from the top margin to the baseline of the header
	%showframe, % Uncomment to show how the type block is set on the page
}

%----------------------------------------------------------------------------------------
%	FONTS
%----------------------------------------------------------------------------------------

\usepackage[utf8]{inputenc} % Required for inputting international characters
\usepackage[T1]{fontenc} % Use 8-bit encoding

% \usepackage{fourier} % Use the Adobe Utopia font for the document

%----------------------------------------------------------------------------------------
%	SECTION TITLES
%----------------------------------------------------------------------------------------

\usepackage{sectsty} % Allows customising section commands

\sectionfont{\vspace{6pt}\centering\normalfont\scshape} % \section{} styling
\subsectionfont{\normalfont\bfseries} % \subsection{} styling
\subsubsectionfont{\normalfont\bfseries} % \subsubsection{} styling
\paragraphfont{\normalfont\scshape} % \paragraph{} styling

%----------------------------------------------------------------------------------------
%	HEADERS AND FOOTERS
%----------------------------------------------------------------------------------------

\usepackage{scrlayer-scrpage} % Required for customising headers and footers

\ohead*{} % Right header
\ihead*{} % Left header
\chead*{} % Centre header

\ofoot*{} % Right footer
\ifoot*{} % Left footer
\cfoot*{\pagemark} % Centre footer

%-----------------------------------------------------------
%  Define new commands
%-----------------------------------------------------------
\DeclareMathOperator*{\argmax}{arg\,max}
\DeclareMathOperator*{\argmin}{arg\,min}
\DeclareMathOperator\supp{supp}

\newcommand*\Let[2]{\State #1 $\gets$ #2}

\DeclareMathOperator*{\E}{\mathbbm{E}}
\DeclareMathOperator*{\Z}{\mathbbm{Z}}
\DeclareMathOperator*{\R}{\mathbbm{R}}

\newtheorem{theorem}{Theorem} % Include the file specifying the document structure and custom commands

%----------------------------------------------------------------------------------------
%	TITLE SECTION
%----------------------------------------------------------------------------------------

\title{
	\normalfont\normalsize
	\textsc{Harvard Privacy Tools Project}\\ % Your university, school and/or department name(s)
	\vspace{25pt} % Whitespace
	\rule{\linewidth}{0.5pt}\\ % Thin top horizontal rule
	\vspace{20pt} % Whitespace
	{\huge Randomness and Noise}\\ % The assignment title
	\vspace{12pt} % Whitespace
	\rule{\linewidth}{2pt}\\ % Thick bottom horizontal rule
	\vspace{12pt} % Whitespace
}

\author{} % Your name

\date{} % Today's date (\today) or a custom date

\begin{document}

\maketitle

\tableofcontents

\section{Overview}
This document describes the strategies the library uses for generation of randomness and noise.

\section{Preliminaries}
\begin{definition}
	\label{defn:differential_privacy}
	Differential Privacy \cite{DMNS06} \newline
	For $\epsilon, \delta \geq 0$, a randomized mechanism
	$\mathcal{M}: \mathcal{X}^n \times \mathcal{Q} \rightarrow \mathcal{Y}$ is
	$(\epsilon, \delta)$-DP if, for every pair of neighboring data sets $X, X' \in \mathcal{X}^n$ and
	every query $q \in \mathcal{Q}$ we have
	\[ \forall \mathcal{T} \subseteq \mathcal{Y}: \Pr[\mathcal{M}(X, \epsilon, \delta, q) \in \mathcal{T}] \leq e^{\epsilon} \Pr[\mathcal{M}(X', \epsilon, \delta, q) \in \mathcal{T}] + \delta. \]
\end{definition}
If $\delta = 0$, we call this \emph{Pure DP}. If $\delta > 0$, we call this \emph{Approximate DP}.
Note that, in practice, differential privacy could be thought of a bit more broadly -- as a bounded
distance between joint distributions over all entities that could be observable to an adversary.\footnote{For example, imagine that the 
US government uses $\epsilon = 1$ if the President is in the data and $\epsilon = 10$
if not -- if anyone knew about this rule, the choice of epsilon would leak information not accounted for in the
traditional definition of DP.} The primary focus of the WhiteNoise library is to respect the traditional notion of 
differential privacy as defined in~\ref{defn:differential_privacy}. However, we are also working to
reduce the possibility of information leakage via differences in computational runtime. We will touch on this when relevant 
during throughout the document.

\begin{definition}
	\label{defn:exact_rounding}
	Exact Rounding \newline
	Let $S \subset \R$ be some set.
	Let $\phi: \R^n \rightarrow \R$ be a function on the reals and $\phi_{S}: S^n \rightarrow S$ be its implementation over $S$.
	Then, $\phi_{S}$ respects \emph{exact rounding} for $(\phi, S)$ if
	\[ \forall s \in S: \phi_{S}(s) = round_{S}[\phi(s)], \]
	where $round_{S}(\cdot)$ rounds a real number to a member of $S$ according to some rounding rule.
\end{definition}
For our purposes, we will care only about the case where $S = \F$, the set of IEEE-754 floating-point numbers.

\begin{definition}
	\label{def:truncation_censoring}
	Truncation and Censoring \newline
	Throughout our noise functions, we use the terms ``truncated'' and ``censored''.
	Both are means of bounding the support of the noise distribution, but they are distinct. \newline

	Truncating a distribution simply ignores events outside of the given bounds, so
	all probabilities within the given bounds are scaled up by a constant factor.
	One way to generate a truncated distribution is via rejection sampling.
	You can generate samples from a probability distribution as you normally would (without any bounding),
	and reject any sample that falls outside of your bounds. \newline

	Censoring a distribution, rather than ignoring events outside of the given bounds, pushes the
	probabilities of said events to the closest event within the given bounds. One way to generate
	a censored distribution would be to generate samples from a probability distribution as you
	typically would, and then clamp samples that fall outside of your bounds to the closest element
	inside your bounds.
\end{definition}

\section{Randomness}
Nearly all of our random number generation involves uniform random sampling of bits via OpenSSL. 
The exception is when we use methods from MPFR (see below) to sample from a probability distribution.
In this case, we feed the state of then OpenSSL RNG directly into the method. \newline 

We will take as given that OpenSSL is cryptographically secure. We intend to support a broader set 
of cryptographically secure sources of randomness at a later date.

\section{Noise Generation}

\subsection{Introduction to MPFR}
The \href{https://www.mpfr.org/}{GNU MPFR Library}\cite{FHL+07} is a C library with methods for carrying out a number
of floating-point operations with \emph{exact rounding}(see Definition~\ref{defn:exact_rounding}).
Among these are basic arithmetic operations and means of generating samples from basic probability distributions. \newline 

\subsection{Biased bit sampling}
Recall that we are taking as given that we are able to sample uniform bits from OpenSSL.
For many applications, however, we want to be able to sample bits non-uniformly,
i.e. where $\Pr(bit = 1) \neq \frac{1}{2}$. To do so, we use the $sample\_bit$ function.
This function uses the unbiased bit generation from OpenSSL to return a single bit, where $\Pr(bit = 1) = prob$ --
there is a nice write-up of the algorithm \href{https://amakelov.wordpress.com/2013/10/10/arbitrarily-biasing-a-coin-in-2-expected-tosses/}{here}.
We will give a general form of the algorithm, and then talk about implementation details.
\begin{algorithm}[H]
	\caption{Biasing an unbiased coin (in theory)}
	\label{alg:biasing_a_coin_theory}
	\begin{algorithmic}[1]
		\State $p \gets \Pr(bit = 1)$
		\State Find the infinite binary expansion of $p$, which we call $b = (b_1, b_2, \hdots,)_2$.
		Note that $p = \sum_{i=1}^{\infty}\frac{b_i}{2^i}$.
		\State Toss an unbiased coin until the first instance of ``heads''. Call the (1-based) index where this occurred $k$.
		\State return $b_k$
	\end{algorithmic}
\end{algorithm}
Let's first show that this procedure gives the correct expectation:
\begin{align*}
	p &= \Pr(bit = 1) \\
		 &= \sum_{i=1}^{\infty} \Pr(bit = 1 \vert k = i) \Pr(k = i) \\
		 &= \sum_{i=1}^{\infty} b_i \cdot \frac{1}{2^i} \\
		 &= \sum_{i=1}^{\infty}\frac{b_i}{2^i}.
\end{align*}
This is consistent with the statement in Algorithm~\ref{alg:biasing_a_coin_theory}, so we know that
the process returns bits with the correct bias.
In terms of efficiency, we know that we can stop coin flipping once we get a heads,
so that part of the algorithm has $\E(\# flips) = 2$ because the probability of getting a heads 
on any given flip is $1/2$. \newline

We now move to constructing the infinite binary expansion of $p$.
We start by noting that, for our purposes, we do not actually need an infinite binary expansion.
Because $p$ will always be a 64-bit floating-point number, we need only get a binary
expansion that covers all representable numbers in our floating-point standard that are
also valid probabilities.
Luckily, the underlying structure of floating-point numbers makes this quite easy. \newline

In the 64-bit standard, floating-point numbers are represented as
\[ (-1)^s(1.m_1m_2 \hdots m_{52})_2 * 2^{(e_{1}e_2 \hdots e_{11})_2 - 1023}, \]
where $s$ is a sign bit we ignore for our purposes.
Our binary expansion is just the mantissa $(1.m_1m_2 \hdots m_{52})_2$, with
the radix point shifted based on the value of the exponent.
We can then index into the properly shifted mantissa and check the value of the $k$th element.
We end up with the following algorithm:
\begin{algorithm}[H]
	\caption{Biasing an unbiased coin (in practice): \newline sample\_bit(p: f64)}
	\label{alg:biasing_a_coin_practice}
	\begin{algorithmic}[1]
		\State We know that $p$ is respresentable as an IEEE-754 64-bit floating point number.
		\State $m, x \gets $ mantissa and exponent of the floating-point representation of $p$. 
		We ensure that the mantissa gets the implicit leading 1 and the exponent is the ``unbiased'' version.
		So $m \in \{1\} \bigcup \{0, 1\}^{52}$ and $x \in \{0, 1, \hdots, 2047\}$
		\State Toss an unbiased coin until the first instance of ``heads''. Call the (0-based) index where this occurred $k$.
		\State n\_leading\_zeros $\gets \max(0, 1022 - x)$
		\If{$k <$ n\_leading\_zeros}
			\State return 0
		\Else
			\State $i \gets$ n\_leading\_zeros $+ k$
			\State return $i^{th}$ element of $m$ (using 0-based indexing)
		\EndIf
	\end{algorithmic}
\end{algorithm}

\subsection{Sampling from censored Geometric}
The Geometric distribution is a building block for many of our other mechanism, either as the basis of the 
noise distribution (as for the Geometric mechanism) or as a component of a larger algorithm (as we will show in 
section~\ref{subsec:unif_sampling}). For now, the library supports sampling only from a censored Geometric distribution. \newline 

The function accepts three arguments. $p$ is the probability of success on any given trial. max\_trials is the maximum number of 
trials for which the algorithm will run -- if no success has occurred by the time max\_trials has been reached, the algorithm 
will return max\_trials as its answer. 
The $ect$ boolean stands for \emph{enforce constant time} and tells the algorithm to run for the maximum number 
of trials, regardless of when success is achieved. This is useful for reducing variability in algorithm runtime that could be used 
for a timing attack. 
\begin{algorithm}[H]
	\caption{Generating draws from Censored Geometric: \newline sample\_geometric\_censored(p: f64, max\_trials: i64, ect: bool)}
	\label{alg:censored_geom_p}
	\begin{algorithmic}[1]
		\State trial\_index $\gets 1$
		\State geom\_return $\gets 0$
		\While {trial\_index $\leq$ max\_trials}
			\State bit $\gets sample\_bit(p)$
			\If{bit $== 1$}
				\If{geom\_return $== 0$} \Comment{Update result from Geometric only if we have not already seen a 1}
					\State geom\_return $\gets$ trial\_index
					\If{ect $==$ False} \Comment{If we do not care to enforce constant time...}
						\State return geom\_return \Comment{return the result}
					\EndIf
				\EndIf
			\Else
				\State trial\_index $+= 1$
			\EndIf
		\EndWhile
		\If{geom\_return $== 0$} \Comment{If Geometric result $>$ censoring bound...}
			\State return max\_trials \Comment{have it return the value of the bound}
		\EndIf
	\end{algorithmic}
\end{algorithm}

\subsection{Sampling from Uniform[min, max)}
\label{subsec:unif_sampling}
In this method, we start by generating a floating-point number $y \in [0,1)$,
where each is generated with probability relative to its unit of least precision (ULP).\footnote{The ULP is the value
represented by the least significant bit of the mantissa if that bit is a 1.} 
That is, we generate $y \in [2^{-g}, 2^{-g+1})$ with probability $\frac{1}{2^i}$
for all $g \in \{1,2,\hdots,1022\}$ and $y \in [0, 2^{-1022})$ for $g = 1023$.
At the end, we will scale our output from $[0,1)$ to be instead in $[min, max)$. \newline 

The algorithm is as follows:
\begin{algorithm}[H]
	\caption{Generating draws from Uniform[min, max)}
	\label{alg:unif_min_max}
	\begin{algorithmic}[1]
		\State $m \gets \{0, 1\}^{52}$ from OpenSSL (or other cryptographically-secure RNG)
		\State $g \gets min(1023, sample\_geometric\_censored(p = 0.5, max\_trials = 1023, ect = True))$
		\State $u \gets (1.m_1m_2 \hdots m_{52})_2 * 2^{-g} * (max - min) + min$
		\State return $u$
	\end{algorithmic}
\end{algorithm}

This method was proposed in \cite{Mir12} as a component of a larger attempt to create
a version of the Laplace mechanism that is not susceptible to floating-point attacks.
Note that the original method generates values $\in [0,1)$ rather than arbitrary $[min, max)$ 
and does not give guidance on what to do if the sample from the Geometric is $> 1023$.
There is no universally agreed upon method for generating uniform random numbers (for privacy
applications or otherwise), but this method seems to approximate the real numbers better than many
others because of the sampling relative to the ULP. \newline

\begin{tcolorbox}[colback = {green}, title = {Known Privacy Issues}, colbacktitle = black]
	When $g=1023$ we are sampling from subnormal floating-point numbers. Because processors do not typically support
	subnormals natively, they take much longer to sample and open us up to an easier timing attack, as
	seen in \cite{AKM+15}. \newline

	We are incurring some floating-point error when converting from $[0,1)$ to $[min, max)$ which
	could jeopardize privacy guarantees in ways that are difficult to reason about.\cite{Mir12} \cite{Ilv19}
\end{tcolorbox}

We have a method for generating uniform samples via MPFR that respects exact rounding, but it is being used sparingly in the library.
We are working to figure out if and how we can use this method as a building block for floating-point safe methods of drawing from 
other distributions.


\subsection{Problems with sampling from other continuous distributions}
In principle, we can generate draws from non-uniform continuous distributions (e.g. Laplace, Gaussian)
by using \href{https://en.wikipedia.org/wiki/Inverse_transform_sampling}{inverse transform sampling}.
To draw from a distribution $f$ with CDF $F$,
we sample $u$ from $Unif[0,1)$ and return $F^{-1}(u)$. \newline

\begin{tcolorbox}[colback = {green}, title = {Known Privacy Issues}, colbacktitle = black]
	Carrying out the inverse probability transform employs floating-point arithmetic,
	so we run into the same problems as were described in the uniform sampling section.
	This is potentially a very significant problem, and one for which we do not
	currently have a good general solution.
\end{tcolorbox}
Because of the vulnerabilities inherent in using floating-point arithmetic, we would
like to avoid using inverse transform sampling. We do not have a completely general way of ensuring 
that algorithms designed to be private when drawing noise from $\R$ remain private when they have 
access only to floating-point numbers. Instead, we will take each relevant distribution individually 
and discuss potential solutions. \newline 

\subsubsection{Potential fixes for Laplace}
We have a \href{https://github.com/opendifferentialprivacy/whitenoise-core/tree/CC_add_snapping}{branch} with an implementation of 
the Snapping mechanism from \cite{Mir12}. We are currently working to verify the theory and associated implementation, as well 
as consider how to use it effectively in practice. \newline 

\subsubsection{Potential fixes for Gaussian}
As mentioned for the Uniform, we have a method for generating Gaussian samples via MPFR that respects exact rounding. 
It is not currently being used in the library. \newline 

\cite{DKM+06} proposes using the binomial approximation to the Gaussian, and notes that
an additive noise mechanism drawing noise from a $Binomial(n, p = 0.5)$ respects $(\epsilon, \delta)$-DP
provided that $n \geq 64 \frac{\log(2/\delta)}{\epsilon^2}$. This seems promising and we already 
have the infrastructure for generating a Binomial without floating-point arithmetic, but it 
involves manually sampling bits. For small $\epsilon$, this method could quickly become 
computationally unwieldy. For example, having the mechanism respect $(10^{-2}, 10^{-9})$-DP
requires sampling nearly 14 million bits.

\bibliographystyle{alpha}
\bibliography{noise}

\end{document}