\documentclass[11pt]{scrartcl} % Font size
%%%%%%%%%%%%%%%%%%%%%%%%%%%%%%%%%%%%%%%%%
% Wenneker Assignment
% Structure Specification File
% Version 2.0 (12/1/2019)
%
% This template originates from:
% http://www.LaTeXTemplates.com
%
% Authors:
% Vel (vel@LaTeXTemplates.com)
% Frits Wenneker
%
% License:
% CC BY-NC-SA 3.0 (http://creativecommons.org/licenses/by-nc-sa/3.0/)
%
%%%%%%%%%%%%%%%%%%%%%%%%%%%%%%%%%%%%%%%%%

%----------------------------------------------------------------------------------------
%	PACKAGES AND OTHER DOCUMENT CONFIGURATIONS
%----------------------------------------------------------------------------------------

\usepackage{amsmath, amsfonts, amsthm} % Math packages
\usepackage{hyperref}
\usepackage{comment}
\usepackage{amsmath}
\usepackage{amssymb}
\usepackage{bbm}
\usepackage{algpseudocode, algorithm, algorithmicx}
\usepackage[toc,page]{appendix}
\usepackage[user,titleref]{zref}

\hypersetup{
   colorlinks=true,
   citecolor=red,
   linkcolor=cyan,
   filecolor=magenta,
   urlcolor=blue,
 }

\usepackage{url}
\usepackage{listings} % Code listings, with syntax highlighting

\usepackage[english]{babel} % English language hyphenation

\usepackage{graphicx} % Required for inserting images
\graphicspath{{Figures/}{./}} % Specifies where to look for included images (trailing slash required)

\usepackage{booktabs} % Required for better horizontal rules in tables

\numberwithin{equation}{section} % Number equations within sections (i.e. 1.1, 1.2, 2.1, 2.2 instead of 1, 2, 3, 4)
\numberwithin{figure}{section} % Number figures within sections (i.e. 1.1, 1.2, 2.1, 2.2 instead of 1, 2, 3, 4)
\numberwithin{table}{section} % Number tables within sections (i.e. 1.1, 1.2, 2.1, 2.2 instead of 1, 2, 3, 4)

\setlength\parindent{0pt} % Removes all indentation from paragraphs

\usepackage{enumitem} % Required for list customisation
\setlist{noitemsep} % No spacing between list items

%----------------------------------------------------------------------------------------
%	DOCUMENT MARGINS
%----------------------------------------------------------------------------------------

\usepackage{geometry} % Required for adjusting page dimensions and margins

\geometry{
	paper=a4paper, % Paper size, change to letterpaper for US letter size
	top=2.5cm, % Top margin
	bottom=3cm, % Bottom margin
	left=3cm, % Left margin
	right=3cm, % Right margin
	headheight=0.75cm, % Header height
	footskip=1.5cm, % Space from the bottom margin to the baseline of the footer
	headsep=0.75cm, % Space from the top margin to the baseline of the header
	%showframe, % Uncomment to show how the type block is set on the page
}

%----------------------------------------------------------------------------------------
%	FONTS
%----------------------------------------------------------------------------------------

\usepackage[utf8]{inputenc} % Required for inputting international characters
\usepackage[T1]{fontenc} % Use 8-bit encoding

% \usepackage{fourier} % Use the Adobe Utopia font for the document

%----------------------------------------------------------------------------------------
%	SECTION TITLES
%----------------------------------------------------------------------------------------

\usepackage{sectsty} % Allows customising section commands

\sectionfont{\vspace{6pt}\centering\normalfont\scshape} % \section{} styling
\subsectionfont{\normalfont\bfseries} % \subsection{} styling
\subsubsectionfont{\normalfont\bfseries} % \subsubsection{} styling
\paragraphfont{\normalfont\scshape} % \paragraph{} styling

%----------------------------------------------------------------------------------------
%	HEADERS AND FOOTERS
%----------------------------------------------------------------------------------------

\usepackage{scrlayer-scrpage} % Required for customising headers and footers

\ohead*{} % Right header
\ihead*{} % Left header
\chead*{} % Centre header

\ofoot*{} % Right footer
\ifoot*{} % Left footer
\cfoot*{\pagemark} % Centre footer

%-----------------------------------------------------------
%  Define new commands
%-----------------------------------------------------------
\DeclareMathOperator*{\argmax}{arg\,max}
\DeclareMathOperator*{\argmin}{arg\,min}
\DeclareMathOperator\supp{supp}

\newcommand*\Let[2]{\State #1 $\gets$ #2}

\DeclareMathOperator*{\E}{\mathbbm{E}}
\DeclareMathOperator*{\Z}{\mathbbm{Z}}
\DeclareMathOperator*{\R}{\mathbbm{R}}

\newtheorem{theorem}{Theorem} % Include the file specifying the document structure and custom commands

%----------------------------------------------------------------------------------------
%	TITLE SECTION
%----------------------------------------------------------------------------------------

\title{
	\normalfont\normalsize
	\textsc{Harvard Privacy Tools Project}\\ % Your university, school and/or department name(s)
	\vspace{25pt} % Whitespace
	\rule{\linewidth}{0.5pt}\\ % Thin top horizontal rule
	\vspace{20pt} % Whitespace
	{\huge Generalized Resize Notes}\\ % The assignment title
	\vspace{12pt} % Whitespace
	\rule{\linewidth}{2pt}\\ % Thick bottom horizontal rule
	\vspace{12pt} % Whitespace
}

\author{} % Your name

\date{} % Today's date (\today) or a custom date

\begin{document}

\maketitle

\textbf{This is a work in progress.}

\section{Goals}
The generalized resize component is a means of jointly achieving a few different goals:
\begin{enumerate}
    \item guarantee known $n$,
    \item give users flexibility in how they trade off between bias and privacy usage, and
    \item allow for a combination of c-stability and privacy amplification from subsampling.
\end{enumerate}

\section{Algorithm Statement}
\label{sec:algorithm_statement}
The function will take the following inputs:
\begin{enumerate}
    \item $X$: The private underlying data.
    \item $\tilde{n}$: The size of the private underlying data.
    \item $n$: The desired size of the new data.
    \item $p$: The proportion of the underlying data that can be used to construct the new data. Can be $> 1$.
    \item $...$: Various arguments explaining imputation rules (not of interest for this doc)
\end{enumerate}

Let $sample(Y, m)$ be a function that samples $m$ elements from data set $Y$ without replacement. 
Let $Aug(Y, m, \hdots)$ be a function that imputes new elements independent of the data (using imputation parameters given by $\hdots$) for a data set $Y$ until it is of size $m$. 
The algorithm will look something like the following:
\begin{algorithm}[H]
    \caption{Generalized Resize: resize(X, n, p, neighboring, ...)}
    \label{alg:gen_resize}
    \begin{algorithmic}[1]
        \State $c \gets \lceil p \rceil$ \Comment{sets c-stability property}
        \State $s \gets p/c$ \Comment{sets subsampled\_proportion property}
        \State $X_c \gets \bigcup_{i=1}^{c}X$ \Comment{create new database of size $c\tilde{n}$, composed of $c$ copies of $X$}
        \If{neighboring == ``replace one''}
            \State $m \gets \lfloor sc\tilde{n} \rfloor$ \Comment{number of records that can be filled using subsampled private data}
        \ElsIf{neighboring == ``add/remove one''}
            \State $m \gets Binomial(c\tilde{n}, s)$ \Comment{number of records that can be filled using subsampled private data}
        \EndIf
        \State $(\epsilon', \delta') \gets \left(\log\left(1+s\left(e^{c\epsilon}-1\right) \right), s\left(\sum_{i=1}^{c-1}e^{i \epsilon}\right)\delta \right)$ \Comment{privacy amplification via subsampling}
        \State $X' \gets sample\left( X_c, \max(m, n), neighboring \right) \bigcup \left( Aug(\varnothing, \max(0, n - m), \hdots) \right)$
        \\ \Return $(X', \epsilon', \delta')$
    \end{algorithmic}
\end{algorithm}

The $\epsilon', \delta'$ terms come from first applying the group privacy definition with group size $c$ to the 
database $X_c$ to get $\left(c\epsilon, \left(\sum_{i=1}^{c-1}e^{i \epsilon}\right)\delta \right)$ \cite{Vad17}
and then applying privacy amplification by subsampling results from Theorems 8 and 9 of \cite{BBG18}. 
Note that, for the ``replace one'' definition, we could be using $\frac{m}{c\tilde{n}}$ instead of $s$ in the privacy calculation. Using $s$ gives us a very slightly 
worse privacy guarantee (the only difference is the $\lfloor \cdot \rfloor$ we used to get $m$), but is nice for 
consistency between the methods and not having to keep track of $m$ as an extra property.

\section{Functional Privacy Parameters}
We established in section~\ref{sec:algorithm_statement} that a user asking for an $(\epsilon, \delta)$-DP guarantee will get an 
$(\epsilon', \delta')$-DP guarantee with respect to the original private data. What we'd really like, however, is for the user to ask for an $(\epsilon, \delta)$-DP 
gurantee and have the library come up with what we will call a \emph{functional $(\epsilon'', \delta'')$} that will ensure $(\epsilon, \delta)$-DP 
on the original data. Any components that operate on the resized data will use the \emph{functional $(\epsilon'', \delta'')$} internally instead of 
the parameters passed by the user.

\begin{theorem}
    A mechanism that respects 
    \[ \left( \frac{1}{c}\log\left(\frac{e^{\epsilon'}-1}{s} + 1 \right), \frac{\delta'}{s\left(\sum_{i=1}^{c-1}e^{i \epsilon}\right)} \right)\text{-DP} \] 
    on the resized data respects $(\epsilon, \delta)$-DP on the true private data.
    \begin{proof}
        We know that an $(\epsilon, \delta)$-DP on the resized data corresponds to a 
        \[ (\epsilon', \delta') = \left(\log\left(1+s\left(e^{c\epsilon}-1\right) \right), s\left(\sum_{i=1}^{c-1}e^{i \epsilon}\right)\delta \right) \]
        guarantee on the private data. We just need to invert the function to find $(\epsilon, \delta)$ in terms of $(\epsilon', \delta')$. \newline 

        Let's start with $\epsilon, \epsilon'$:
        \begin{align*}
            \epsilon' &= \log\left(1+s\left(e^{c\epsilon}-1\right)\right) \\
            e^{\epsilon'} &= 1+s\left(e^{c\epsilon}-1\right) \\
            \frac{e^{\epsilon'}-1}{s} &= e^{c\epsilon}-1 \\
            \frac{1}{c}\log\left(\frac{e^{\epsilon'}-1}{s} + 1 \right) &= \epsilon.
        \end{align*}

        We carry out a similar calculation for $\delta, \delta'$:
        \begin{align*}
            \delta' &= s\left(\sum_{i=1}^{c-1}e^{i \epsilon}\right)\delta \\
            \frac{\delta'}{s\left(\sum_{i=1}^{c-1}e^{i \epsilon}\right)} &= \delta.
        \end{align*}
    \end{proof} 
    So, in order for a mechanism to respect $(\epsilon, \delta)$-DP on the original data, it must respect 
    $\left( \frac{1}{c}\log\left(\frac{e^{\epsilon'}-1}{s} + 1 \right), \frac{\delta'}{s\left(\sum_{i=1}^{c-1}e^{i \epsilon}\right)} \right)$-DP 
    on the resized data.
\end{theorem}

We now present an mini-algorithm for finding the \emph{functional} $(\epsilon'', \delta'')$.
\begin{algorithm}
    \caption{Finding Functional $(\epsilon'', \delta'')$: get\_func\_priv($p, \epsilon, \delta$)}
    \label{alg:finding_functional_privacy}
    \begin{algorithmic}[1]
        \State $c \gets \lceil p \rceil$ \Comment{sets c-stability property}
        \State $s \gets p/c$ \Comment{sets subsampled\_proportion property}
        \State $\epsilon'' \gets \frac{1}{c}\log\left(\frac{e^{\epsilon'}-1}{s} + 1\right)$
        \State $\delta'' \gets \frac{\delta'}{s\left(\sum_{i=1}^{c-1}e^{i \epsilon}\right)}$
        \\ \Return $(\epsilon'', \delta'')$ 
    \end{algorithmic}
\end{algorithm}

\section{Examples} \label{sec:examples}
Let $X$ be such that $\tilde{n} = \vert X \vert = 100$. We can look at a few examples of calls to the generalized resize function (which will return $X'$) and check the behavior.
\begin{enumerate}
    \item resize(X, 150, 1, ...): $X'$ will be made up of the 100 true elements of $X$ and 50 imputed values. The functional privacy parameters are identical to the ones the user provides. 
    \item resize(X, 100, 0.75, ...): $X'$ will be made up of 75 true elements of $X$ and 25 imputed values. The functional privacy parameters will benefit (lower noise) from amplification via subsampling.
    \item resize(X, 90, 1.5, ...): $X'$ will be a random sample of $X \bigcup X$ of size 90. 
            The functional privacy parameters will lead to greater noise than what the user provides, as they have to take into account the new c-stability of $2$. 
            This example is illustrative in that is shows that the functional privacy usage is affected only by $p$ -- it has nothing to do with the relative sizes of the $X$ and $X'$.
\end{enumerate}

\bibliographystyle{alpha}
\bibliography{generalized_resize}

\end{document}
